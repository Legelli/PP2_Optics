% !TEX TS-program = XeLaTeX
%!TEX encoding = UTF-8 Unicode
%==================================================
%      PREAMBOLO e DICHIARAZIONI INIZIALI
%==================================================
\documentclass[10pt,oneside,a4paper]{article}

\usepackage[utf8]{inputenc} 
\usepackage[italian]{babel}
\usepackage[T1]{fontenc}
\usepackage{siunitx} %Inserisce automaticamente i dati con le unità  di misura correttamente formattate del SI (utilizzo: \SI{0.82}{m^2}, in generale \SI{misura con il punto decimale}{unità  di misura})
\sisetup{output-decimal-marker = {.}, separate-uncertainty = true, input-uncertainty-signs = \pm, detect-weight=true, detect-family=true} %per usare SI con il punto decimale
\usepackage{listings} %Per citare codice informatico formattandolo correttamente
\usepackage{amsmath,amsthm,verbatim,amssymb,amsfonts,amscd,graphicx,mathtools}
\usepackage[makeroom]{cancel}
\newcommand{\abs}[1]{\left\lvert\,#1\,\right\rvert}
\usepackage{geometry}
\usepackage{epigraph}
\usepackage{booktabs}	%tabelle migliorate
\usepackage{tablefootnote}	%note a piè di pagina in tabella
\usepackage{threeparttable} %tabella con note a piè di tabella
\usepackage{caption}	%descrizione per figure
\usepackage{dblfnote}
\captionsetup{tableposition=top,figureposition=bottom,font=small} %setup descrizione
\usepackage{float}
\usepackage{esvect} %vettori
\usepackage{longtable} %tabelle lunghe
\usepackage[dvipsnames]{xcolor}
\definecolor{sepia}{HTML}{80002A}
\usepackage[colorlinks=true, citecolor=black, linkcolor=sepia, urlcolor=black]{hyperref}
\usepackage{mathrsfs}
\usepackage{circuitikz}
\tikzset{
  font={\fontsize{7pt}{12}\selectfont}}
\ctikzset{bipoles/resistor/height=0.2}
\ctikzset{bipoles/resistor/width=0.4}
\ctikzset{bipoles/diode/height=0.3}
\ctikzset{bipoles/diode/width=0.3}
\ctikzset{tripoles/american nand port/height=0.7}
\ctikzset{tripoles/american nand port/width=0.8}
\usepackage{enumitem} %Liste senza spazi verticali
\setlist{noitemsep}
\usepackage{amsmath}
\usepackage{hyperref}
%\usepackage{pst-optexp} %Diagrammi ottici
\usepackage{physics} %Ambienti utili


\interfootnotelinepenalty=10000


\usepackage{multicol}
\newenvironment{Figure}
  {\par\medskip\noindent\minipage{\linewidth}}
  {\endminipage\par\medskip}

%\newcommand{\var}{\operatorname{var}}
%\newcommand{\cov}{\operatorname{cov}}


\usepackage{listings} %Per inserire codice
\lstdefinestyle{CStyle}{
    backgroundcolor=\color{backgroundColour},   
    commentstyle=\color{mGreen},
    keywordstyle=\color{magenta},
    numberstyle=\tiny\color{mGray},
    stringstyle=\color{mPurple},
    basicstyle=\footnotesize\ttfamily,
    breakatwhitespace=false,         
    breaklines=true,                 
    captionpos=b,                    
    keepspaces=true,                 
    numbers=left,                    
    numbersep=5pt,                  
    showspaces=false,                
    showstringspaces=false,
    showtabs=false,                  
    tabsize=2,
    language=C
}

\definecolor{color1}{RGB}{90,0,0} % Color of the article title and sections
\definecolor{color2}{RGB}{0,20,50} % Color of the boxes behind the abstract and headings
\definecolor{mGreen}{rgb}{0,0.6,0}
\definecolor{mGray}{rgb}{0.5,0.5,0.5}
\definecolor{mPurple}{rgb}{0.58,0,0.82}
\definecolor{backgroundColour}{rgb}{0.95,0.95,0.92}


%==================================================
%                  PRIMA PAGINA
%==================================================

\title{\textsc{\textbf{Esperienza 2}: Interferometro di Michelson}}
\author{\small{G. Galbato Muscio} \and \small{F. Ghimenti} \and \small{L. Gravina} \and \small{L. Graziotto}}
\date{11 Aprile 2019}

\begin{document}
	\begin{figure}
		\centering
		\includegraphics[scale=0.5, trim={2.8cm 8.9cm 0 9cm}, clip]{logo.png}
	\end{figure}
	\maketitle
	\begin{center} 
		\fbox{{\fontsize{12pt}{8mm}\textsc{Gruppo D1-1}}} \\
	\end{center}
\hrule
\vfill
\renewcommand{\abstractname}{Abstract}
\begin{abstract}
Si studia la visibilità delle frange di interferenza di un laser HeNe\footnote{\url{https://www.dropbox.com/s/5aqzs2uykfi8lms/8-Coherence_function_He_Ne.pdf?dl=0}} mediante un interferometro di Michelson e si misura il tempo di correlazione $\tau_\mathrm{c}$ del laser.
\end{abstract}
\vfill
\tableofcontents %Indice
\newpage


\pagebreak


\begin{multicols}{2}
%==================================================
%             APPARATO STRUMENTALE
%==================================================
\section{Apparato strumentale}

Si utilizza un laser He-Ne di lunghezza d'onda $\lambda = \SI{632.8}{nm}$, montato su tavolo ottico. 

In serie al laser è posta un'iride, con lo scopo di evitare fasci entranti nel laser e perturbarne il comportamento. Uno specchio riflette la luce uscente dell'iride in un beam splitter, i due fasci ortogonali uscenti vengono fatti riflettere su due specchi e dunque ricombinati all'interno del beam splitter, una lente divergente allarga uno dei due fasci ricombinati prima di essere misurato da un fotodiodo.

Uno dei due specchi costituenti l'interferometro viene installato su una base fissa e appoggiato su un cristallo piezoelettrico, quest'ultimo è fatto espandere e contrarre attraverso un'onda triangolare di frequenza $f_\mathrm{rampa} = \SI{1.016}{\hertz}$ e ampiezza $A_\mathrm{rampa} = \SI{20.8}{\volt}$; l'altro viene fatto traslare su un piano forato con distanza trai fori pari a $d = \SI{2.5 \pm 0.1}{cm}$. 

La configurazione utilizzata è illustrata in Figura~\ref{fig:diagram}.

Il segnale in uscita dal fotodiodo è misurato con un oscilloscopio, quest'ultimo viene sincronizzato sulla frequenza del segnale uscente dal generatore.

\section{Visibilità delle frange}
\subsection{Presa dati}
Si posiziona lo specchio mobile a distanza $d^{(0)}_\mathrm{mobile} = \SI{7.3 \pm 0.1}{cm}$, e quello fisso a distanza $d^{(0)}_\mathrm{fisso} = \SI{6.4 \pm 0.1}{cm}$, si muovono gli specchi con l'utilizzo di viti micrometriche al fine di collimare il più possibile i due fasci riflessi per massimizzare la visibilità percepita delle frange di interferenza e centrare il fascio allargato dalla lente sul fotodiodo. Sull'oscilloscopio compare l'intensità del fascio uscente dall'interferometro, fascio che segue un andamento sinusoidale\footnote{L'andamento dell'intensità è coerente con quello di contrazione e dilatazione del piezoelettrico, appunto sinusoidale.}, si misura quindi l'intensità massima $I_\mathrm{max}$ e minima $I_\mathrm{min}$ all'interno di una singola rampa (indifferentemente di salita o di discesa) del segnale che pilota il piezoelettrico. Questa sequenza di operazioni viene ripetuta per tre volte, disallineando manualmente i fasci prima di riallinearli, quindi lo specchio mobile è spostato ad una distanza $d(N) = d^{(0)}_\mathrm{mobile} + Nd$, dove $N$ è il numero di fori liberi contati tra la base del beamsplitter e quella dello specchio mobile, mentre $d$ è la distanza intraforo riportata in precedenza.
L'insieme di misure prese è riportato in Tabella \ref{tab:intensita}.

In seguito alle misure, ne viene effettuata un'ultima riavvicinando lo specchio mobile a distanza $d_\mathrm{mobile} = \SI{5.6 \pm 0.1}{cm}$ dal beamsplitter.

Tutte le misure prese sono riportate in Figura \ref{fig:visibilita_tutte} in funzione della differenza di cammino ottico definita come $\tau \equiv \left\vert c^{-1}\left(d^{(0)}_\mathrm{fisso} - d(N)\right) \right\vert$.
L'incertezza associata alla misura è quella fornita dal manuale\footnote{\url{http://pdf1.alldatasheet.com/datasheet-pdf/view/554089/ETC2/TDS2012C.html}} dell'oscilloscopio, ossia il $3\%$.

\subsection{Analisi}

\section{Conclusioni}


\end{multicols}

\section{Appendice}

%ESEMPIO DI FIGURA
%\begin{Figure}
%	\begin{center}
%	\includegraphics[width=\linewidth]{comune.png}
%	\captionof{figure}{Istantanea dell'oscilloscopio per l'amplificatore differenziale, misura di $A_c$}
%	\label{fig:Ac_differenziale}
%	\end{center}
%\end{Figure}


%ESEMPIO DI TABELLA
%\begin{center}
%\captionof{table}{Misure per la stima di $A_c$}
%\label{tab:Ac_differenziale}
%\begin{tabular}{c|c|c|c}
%$f$ [\SI{}{Hz}] & $V_i$ [\SI{}{V}] & $v_o$ [\SI{}{mV}] & $A_c = v_o / V_i$ \\
%\hline
%      149.5 &        3.90 &         11.3 & 2.90e-03 \\
%      222.0 &        3.90 &         11.5 & 2.95e-03 \\
%      281.0 &        3.90 &         11.8 & 3.03e-03 \\
%      359.0 &        3.90 &         11.8 & 3.03e-03 \\
%      461.0 &        3.90 &         12.2 & 3.13e-03 \\
%\hline
%\end{tabular}
%\end{center}


\end{document}
